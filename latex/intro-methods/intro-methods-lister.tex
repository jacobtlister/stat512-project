% !TEX TS-program = pdflatex
% !TEX encoding = UTF-8 Unicode
\documentclass{article}

\newcommand{\filetitle}{intro-methods-lister}

% packages used
\usepackage{amsmath}
\usepackage{amssymb}
\usepackage{mathtools}
\usepackage{bm}
\usepackage{graphicx}
\usepackage{iclr2025_conference,times}
\usepackage{hyperref}
\usepackage{url}
\usepackage{natbib}
\usepackage{fancyhdr}

% directory where images placed in document live
\graphicspath{ {./figures/} }

% adjusting hyperref options to my liking
\hypersetup{
    colorlinks=true,
    breaklinks=true,
    urlcolor=blue,
    linkcolor=black,
    filecolor=black,
    citecolor=black,
    menucolor=black,
    urlbordercolor={1 1 1},
    linkbordercolor={1 1 1},
    filebordercolor={1 1 1},
    citebordercolor={1 1 1},
    menubordercolor={1 1 1},
    bookmarksopen=true,
    pdfpagemode=useOutlines,
    pdftitle={\filetitle},
    pdfauthor={Jacob Lister},
}

% Optional math commands from https://github.com/goodfeli/dlbook_notation.
\input{math_commands.tex}

\title{Gaussian Processes/Robot Arm Dynamics \\ Introduction and Methods}

\author {Jacob Lister                              \\
Department of Electrical and Computer Engineerning \\
Department of Mathematical Sciences                \\
Purdue University Fort Wayne                       \\
Fort Wayne, IN 46805, USA                          \\
\texttt{aldrjt01@pfw.edu}
}

\newcommand{\fix}{\marginpar{FIX}}
\newcommand{\new}{\marginpar{NEW}}

\iclrfinalcopy % Uncomment for camera-ready version, but NOT for submission.
\begin{document}

    \maketitle

    \begin{abstract}
        abstract
    \end{abstract}

    \section{Introduction}

    intro
    
    \subsection{Problem Overview}
    
    talk about inverse dynamics for robot arms problem statement
    
    \subsection{Kinematics Overview}
    
    robot movement is highly nonlinear (explain kinematics a bit)
    
    \section{Methods}
    
    In this section, I describe the data source, Guassian Processes Regression (GPR) which I am
    considering, Projected Processes (PP) Approximation, and provide a summary of my analysis.
    
    \subsection{Data Source and Software}
    
    The SARCOS dataset was taken from the website for the textbook 
    \textit{Gaussian Processes for Machine Learning} \citet{sarcos}. The analysis will be performed
    using the software Python 3.10 and the packages \texttt{numpy} and \texttt{scipy}. The dataset
    is publicly available and the code will be publicly available in a Github Repository
    \citep{github}.
    
    \subsection{Guassian Processes Regression (GPR)}
    
    gaussian process regression
    
    \subsection{Projected Processes Approximation}
    
    Use the Project Processes (PP) methods due to combination of high performance and lower
    time complexity. Has performance very similar to ST and BCM methods
    
    cite table 8.1 for methods section when talking about time comparison for SR and PP methods

    \section{Citations, figures, tables, references}

    These instructions apply to everyone, regardless of the formatter being used.

    \subsection{Citations within the text}

    Citations within the text should be based on the \texttt{natbib} package
    and include the authors' last names and year (with the ``et~al.'' construct
    for more than two authors). When the authors or the publication are
    included in the sentence, the citation should not be in parenthesis using \verb|\citet{}| (as
    in ``See \citet{Hinton06} for more information.''). Otherwise, the citation
    should be in parenthesis using \verb|\citep{}| (as in ``Deep learning shows promise to make progress
    towards AI~\citep{Bengio+chapter2007}.'').

    The corresponding references are to be listed in alphabetical order of
    authors, in the \textsc{References} section. As to the format of the
    references themselves, any style is acceptable as long as it is used
    consistently.

    \subsection{Figures}

    All artwork must be neat, clean, and legible. Lines should be dark
    enough for purposes of reproduction; art work should not be
    hand-drawn. The figure number and caption always appear after the
    figure. Place one line space before the figure caption, and one line
    space after the figure. The figure caption is lower case (except for
    first word and proper nouns); figures are numbered consecutively.

    Make sure the figure caption does not get separated from the figure.
    Leave sufficient space to avoid splitting the figure and figure caption.

    You may use color figures.
    However, it is best for the
    figure captions and the paper body to make sense if the paper is printed
    either in black/white or in color.
    
    \begin{figure}[h]
    \begin{center}
        %\framebox[4.0in]{$\;$}
        \fbox{\rule[-.5cm]{0cm}{4cm} \rule[-.5cm]{4cm}{0cm}}
    \end{center}
        \caption{Sample figure caption.}
    \end{figure}

    \subsection{Tables}

    All tables must be centered, neat, clean and legible. Do not use hand-drawn
    tables. The table number and title always appear before the table. See
    Table~\ref{sample-table}.

    Place one line space before the table title, one line space after the table
    title, and one line space after the table. The table title must be lower case
    (except for first word and proper nouns); tables are numbered consecutively.

    \begin{table}[t]
        \caption{Sample table title}
        \label{sample-table}
    \begin{center}
    \begin{tabular}{ll}
        \multicolumn{1}{c}{\bf PART}  &\multicolumn{1}{c}{\bf DESCRIPTION}
        \\ \hline \\
        Dendrite         &Input terminal \\
        Axon             &Output terminal \\
        Soma             &Cell body (contains cell nucleus) \\
    \end{tabular}
    \end{center}
    \end{table}

\subsection{Margins in LaTeX}

Most of the margin problems come from figures positioned by hand using
\verb+\special+ or other commands. We suggest using the command
\verb+\includegraphics+
from the graphicx package. Always specify the figure width as a multiple of
the line width as in the example below using .eps graphics
\begin{verbatim}
   \usepackage[dvips]{graphicx} ...
   \includegraphics[width=0.8\linewidth]{myfile.eps}
\end{verbatim}
or % Apr 2009 addition
\begin{verbatim}
   \usepackage[pdftex]{graphicx} ...
   \includegraphics[width=0.8\linewidth]{myfile.pdf}
\end{verbatim}
for .pdf graphics.
See section~4.4 in the graphics bundle documentation (\url{http://www.ctan.org/tex-archive/macros/latex/required/graphics/grfguide.ps})

A number of width problems arise when LaTeX cannot properly hyphenate a
line. Please give LaTeX hyphenation hints using the \verb+\-+ command.

    \bibliography{intro-methods}
    \bibliographystyle{iclr2025_conference}

\end{document}
