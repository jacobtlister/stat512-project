% !TEX TS-program = pdflatex
% !TEX encoding = UTF-8 Unicode
\documentclass{article}

\newcommand{\filetitle}{analysis-results-lister}

% packages used
\usepackage{amsmath}
\usepackage{amssymb}
\usepackage{mathtools}
\usepackage{bm}
\usepackage{graphicx}
\usepackage{iclr2025_conference,times}
\usepackage{hyperref}
\usepackage{url}
\usepackage{natbib}
\usepackage{fancyhdr}
\usepackage{booktabs}

% directory where images placed in document live
\graphicspath{ {./figures/} }

% adjusting hyperref options to my liking
\hypersetup{
    colorlinks=true,
    breaklinks=true,
    urlcolor=blue,
    linkcolor=black,
    filecolor=black,
    citecolor=black,
    menucolor=black,
    urlbordercolor={1 1 1},
    linkbordercolor={1 1 1},
    filebordercolor={1 1 1},
    citebordercolor={1 1 1},
    menubordercolor={1 1 1},
    bookmarksopen=true,
    pdfpagemode=useOutlines,
    pdftitle={\filetitle},
    pdfauthor={Jacob Lister},
}

% Optional math commands from https://github.com/goodfeli/dlbook_notation.
%%%%% NEW MATH DEFINITIONS %%%%%

% Mark sections of captions for referring to divisions of figures
\newcommand{\figleft}{{\em (Left)}}
\newcommand{\figcenter}{{\em (Center)}}
\newcommand{\figright}{{\em (Right)}}
\newcommand{\figtop}{{\em (Top)}}
\newcommand{\figbottom}{{\em (Bottom)}}
\newcommand{\captiona}{{\em (a)}}
\newcommand{\captionb}{{\em (b)}}
\newcommand{\captionc}{{\em (c)}}
\newcommand{\captiond}{{\em (d)}}

% Highlight a newly defined term
\newcommand{\newterm}[1]{{\bf #1}}


% Figure reference, lower-case.
\def\figref#1{figure~\ref{#1}}
% Figure reference, capital. For start of sentence
\def\Figref#1{Figure~\ref{#1}}
\def\twofigref#1#2{figures \ref{#1} and \ref{#2}}
\def\quadfigref#1#2#3#4{figures \ref{#1}, \ref{#2}, \ref{#3} and \ref{#4}}
% Section reference, lower-case.
\def\secref#1{section~\ref{#1}}
% Section reference, capital.
\def\Secref#1{Section~\ref{#1}}
% Reference to two sections.
\def\twosecrefs#1#2{sections \ref{#1} and \ref{#2}}
% Reference to three sections.
\def\secrefs#1#2#3{sections \ref{#1}, \ref{#2} and \ref{#3}}
% Reference to an equation, lower-case.
\def\eqref#1{equation~\ref{#1}}
% Reference to an equation, upper case
\def\Eqref#1{Equation~\ref{#1}}
% A raw reference to an equation---avoid using if possible
\def\plaineqref#1{\ref{#1}}
% Reference to a chapter, lower-case.
\def\chapref#1{chapter~\ref{#1}}
% Reference to an equation, upper case.
\def\Chapref#1{Chapter~\ref{#1}}
% Reference to a range of chapters
\def\rangechapref#1#2{chapters\ref{#1}--\ref{#2}}
% Reference to an algorithm, lower-case.
\def\algref#1{algorithm~\ref{#1}}
% Reference to an algorithm, upper case.
\def\Algref#1{Algorithm~\ref{#1}}
\def\twoalgref#1#2{algorithms \ref{#1} and \ref{#2}}
\def\Twoalgref#1#2{Algorithms \ref{#1} and \ref{#2}}
% Reference to a part, lower case
\def\partref#1{part~\ref{#1}}
% Reference to a part, upper case
\def\Partref#1{Part~\ref{#1}}
\def\twopartref#1#2{parts \ref{#1} and \ref{#2}}

\def\ceil#1{\lceil #1 \rceil}
\def\floor#1{\lfloor #1 \rfloor}
\def\1{\bm{1}}
\newcommand{\train}{\mathcal{D}}
\newcommand{\valid}{\mathcal{D_{\mathrm{valid}}}}
\newcommand{\test}{\mathcal{D_{\mathrm{test}}}}

\def\eps{{\epsilon}}


% Random variables
\def\reta{{\textnormal{$\eta$}}}
\def\ra{{\textnormal{a}}}
\def\rb{{\textnormal{b}}}
\def\rc{{\textnormal{c}}}
\def\rd{{\textnormal{d}}}
\def\re{{\textnormal{e}}}
\def\rf{{\textnormal{f}}}
\def\rg{{\textnormal{g}}}
\def\rh{{\textnormal{h}}}
\def\ri{{\textnormal{i}}}
\def\rj{{\textnormal{j}}}
\def\rk{{\textnormal{k}}}
\def\rl{{\textnormal{l}}}
% rm is already a command, just don't name any random variables m
\def\rn{{\textnormal{n}}}
\def\ro{{\textnormal{o}}}
\def\rp{{\textnormal{p}}}
\def\rq{{\textnormal{q}}}
\def\rr{{\textnormal{r}}}
\def\rs{{\textnormal{s}}}
\def\rt{{\textnormal{t}}}
\def\ru{{\textnormal{u}}}
\def\rv{{\textnormal{v}}}
\def\rw{{\textnormal{w}}}
\def\rx{{\textnormal{x}}}
\def\ry{{\textnormal{y}}}
\def\rz{{\textnormal{z}}}

% Random vectors
\def\rvepsilon{{\mathbf{\epsilon}}}
\def\rvtheta{{\mathbf{\theta}}}
\def\rva{{\mathbf{a}}}
\def\rvb{{\mathbf{b}}}
\def\rvc{{\mathbf{c}}}
\def\rvd{{\mathbf{d}}}
\def\rve{{\mathbf{e}}}
\def\rvf{{\mathbf{f}}}
\def\rvg{{\mathbf{g}}}
\def\rvh{{\mathbf{h}}}
\def\rvu{{\mathbf{i}}}
\def\rvj{{\mathbf{j}}}
\def\rvk{{\mathbf{k}}}
\def\rvl{{\mathbf{l}}}
\def\rvm{{\mathbf{m}}}
\def\rvn{{\mathbf{n}}}
\def\rvo{{\mathbf{o}}}
\def\rvp{{\mathbf{p}}}
\def\rvq{{\mathbf{q}}}
\def\rvr{{\mathbf{r}}}
\def\rvs{{\mathbf{s}}}
\def\rvt{{\mathbf{t}}}
\def\rvu{{\mathbf{u}}}
\def\rvv{{\mathbf{v}}}
\def\rvw{{\mathbf{w}}}
\def\rvx{{\mathbf{x}}}
\def\rvy{{\mathbf{y}}}
\def\rvz{{\mathbf{z}}}

% Elements of random vectors
\def\erva{{\textnormal{a}}}
\def\ervb{{\textnormal{b}}}
\def\ervc{{\textnormal{c}}}
\def\ervd{{\textnormal{d}}}
\def\erve{{\textnormal{e}}}
\def\ervf{{\textnormal{f}}}
\def\ervg{{\textnormal{g}}}
\def\ervh{{\textnormal{h}}}
\def\ervi{{\textnormal{i}}}
\def\ervj{{\textnormal{j}}}
\def\ervk{{\textnormal{k}}}
\def\ervl{{\textnormal{l}}}
\def\ervm{{\textnormal{m}}}
\def\ervn{{\textnormal{n}}}
\def\ervo{{\textnormal{o}}}
\def\ervp{{\textnormal{p}}}
\def\ervq{{\textnormal{q}}}
\def\ervr{{\textnormal{r}}}
\def\ervs{{\textnormal{s}}}
\def\ervt{{\textnormal{t}}}
\def\ervu{{\textnormal{u}}}
\def\ervv{{\textnormal{v}}}
\def\ervw{{\textnormal{w}}}
\def\ervx{{\textnormal{x}}}
\def\ervy{{\textnormal{y}}}
\def\ervz{{\textnormal{z}}}

% Random matrices
\def\rmA{{\mathbf{A}}}
\def\rmB{{\mathbf{B}}}
\def\rmC{{\mathbf{C}}}
\def\rmD{{\mathbf{D}}}
\def\rmE{{\mathbf{E}}}
\def\rmF{{\mathbf{F}}}
\def\rmG{{\mathbf{G}}}
\def\rmH{{\mathbf{H}}}
\def\rmI{{\mathbf{I}}}
\def\rmJ{{\mathbf{J}}}
\def\rmK{{\mathbf{K}}}
\def\rmL{{\mathbf{L}}}
\def\rmM{{\mathbf{M}}}
\def\rmN{{\mathbf{N}}}
\def\rmO{{\mathbf{O}}}
\def\rmP{{\mathbf{P}}}
\def\rmQ{{\mathbf{Q}}}
\def\rmR{{\mathbf{R}}}
\def\rmS{{\mathbf{S}}}
\def\rmT{{\mathbf{T}}}
\def\rmU{{\mathbf{U}}}
\def\rmV{{\mathbf{V}}}
\def\rmW{{\mathbf{W}}}
\def\rmX{{\mathbf{X}}}
\def\rmY{{\mathbf{Y}}}
\def\rmZ{{\mathbf{Z}}}

% Elements of random matrices
\def\ermA{{\textnormal{A}}}
\def\ermB{{\textnormal{B}}}
\def\ermC{{\textnormal{C}}}
\def\ermD{{\textnormal{D}}}
\def\ermE{{\textnormal{E}}}
\def\ermF{{\textnormal{F}}}
\def\ermG{{\textnormal{G}}}
\def\ermH{{\textnormal{H}}}
\def\ermI{{\textnormal{I}}}
\def\ermJ{{\textnormal{J}}}
\def\ermK{{\textnormal{K}}}
\def\ermL{{\textnormal{L}}}
\def\ermM{{\textnormal{M}}}
\def\ermN{{\textnormal{N}}}
\def\ermO{{\textnormal{O}}}
\def\ermP{{\textnormal{P}}}
\def\ermQ{{\textnormal{Q}}}
\def\ermR{{\textnormal{R}}}
\def\ermS{{\textnormal{S}}}
\def\ermT{{\textnormal{T}}}
\def\ermU{{\textnormal{U}}}
\def\ermV{{\textnormal{V}}}
\def\ermW{{\textnormal{W}}}
\def\ermX{{\textnormal{X}}}
\def\ermY{{\textnormal{Y}}}
\def\ermZ{{\textnormal{Z}}}

% Vectors
\def\vzero{{\bm{0}}}
\def\vone{{\bm{1}}}
\def\vmu{{\bm{\mu}}}
\def\vtheta{{\bm{\theta}}}
\def\va{{\bm{a}}}
\def\vb{{\bm{b}}}
\def\vc{{\bm{c}}}
\def\vd{{\bm{d}}}
\def\ve{{\bm{e}}}
\def\vf{{\bm{f}}}
\def\vg{{\bm{g}}}
\def\vh{{\bm{h}}}
\def\vi{{\bm{i}}}
\def\vj{{\bm{j}}}
\def\vk{{\bm{k}}}
\def\vl{{\bm{l}}}
\def\vm{{\bm{m}}}
\def\vn{{\bm{n}}}
\def\vo{{\bm{o}}}
\def\vp{{\bm{p}}}
\def\vq{{\bm{q}}}
\def\vr{{\bm{r}}}
\def\vs{{\bm{s}}}
\def\vt{{\bm{t}}}
\def\vu{{\bm{u}}}
\def\vv{{\bm{v}}}
\def\vw{{\bm{w}}}
\def\vx{{\bm{x}}}
\def\vy{{\bm{y}}}
\def\vz{{\bm{z}}}

% Elements of vectors
\def\evalpha{{\alpha}}
\def\evbeta{{\beta}}
\def\evepsilon{{\epsilon}}
\def\evlambda{{\lambda}}
\def\evomega{{\omega}}
\def\evmu{{\mu}}
\def\evpsi{{\psi}}
\def\evsigma{{\sigma}}
\def\evtheta{{\theta}}
\def\eva{{a}}
\def\evb{{b}}
\def\evc{{c}}
\def\evd{{d}}
\def\eve{{e}}
\def\evf{{f}}
\def\evg{{g}}
\def\evh{{h}}
\def\evi{{i}}
\def\evj{{j}}
\def\evk{{k}}
\def\evl{{l}}
\def\evm{{m}}
\def\evn{{n}}
\def\evo{{o}}
\def\evp{{p}}
\def\evq{{q}}
\def\evr{{r}}
\def\evs{{s}}
\def\evt{{t}}
\def\evu{{u}}
\def\evv{{v}}
\def\evw{{w}}
\def\evx{{x}}
\def\evy{{y}}
\def\evz{{z}}

% Matrix
\def\mA{{\bm{A}}}
\def\mB{{\bm{B}}}
\def\mC{{\bm{C}}}
\def\mD{{\bm{D}}}
\def\mE{{\bm{E}}}
\def\mF{{\bm{F}}}
\def\mG{{\bm{G}}}
\def\mH{{\bm{H}}}
\def\mI{{\bm{I}}}
\def\mJ{{\bm{J}}}
\def\mK{{\bm{K}}}
\def\mL{{\bm{L}}}
\def\mM{{\bm{M}}}
\def\mN{{\bm{N}}}
\def\mO{{\bm{O}}}
\def\mP{{\bm{P}}}
\def\mQ{{\bm{Q}}}
\def\mR{{\bm{R}}}
\def\mS{{\bm{S}}}
\def\mT{{\bm{T}}}
\def\mU{{\bm{U}}}
\def\mV{{\bm{V}}}
\def\mW{{\bm{W}}}
\def\mX{{\bm{X}}}
\def\mY{{\bm{Y}}}
\def\mZ{{\bm{Z}}}
\def\mBeta{{\bm{\beta}}}
\def\mPhi{{\bm{\Phi}}}
\def\mLambda{{\bm{\Lambda}}}
\def\mSigma{{\bm{\Sigma}}}

% Tensor
\DeclareMathAlphabet{\mathsfit}{\encodingdefault}{\sfdefault}{m}{sl}
\SetMathAlphabet{\mathsfit}{bold}{\encodingdefault}{\sfdefault}{bx}{n}
\newcommand{\tens}[1]{\bm{\mathsfit{#1}}}
\def\tA{{\tens{A}}}
\def\tB{{\tens{B}}}
\def\tC{{\tens{C}}}
\def\tD{{\tens{D}}}
\def\tE{{\tens{E}}}
\def\tF{{\tens{F}}}
\def\tG{{\tens{G}}}
\def\tH{{\tens{H}}}
\def\tI{{\tens{I}}}
\def\tJ{{\tens{J}}}
\def\tK{{\tens{K}}}
\def\tL{{\tens{L}}}
\def\tM{{\tens{M}}}
\def\tN{{\tens{N}}}
\def\tO{{\tens{O}}}
\def\tP{{\tens{P}}}
\def\tQ{{\tens{Q}}}
\def\tR{{\tens{R}}}
\def\tS{{\tens{S}}}
\def\tT{{\tens{T}}}
\def\tU{{\tens{U}}}
\def\tV{{\tens{V}}}
\def\tW{{\tens{W}}}
\def\tX{{\tens{X}}}
\def\tY{{\tens{Y}}}
\def\tZ{{\tens{Z}}}


% Graph
\def\gA{{\mathcal{A}}}
\def\gB{{\mathcal{B}}}
\def\gC{{\mathcal{C}}}
\def\gD{{\mathcal{D}}}
\def\gE{{\mathcal{E}}}
\def\gF{{\mathcal{F}}}
\def\gG{{\mathcal{G}}}
\def\gH{{\mathcal{H}}}
\def\gI{{\mathcal{I}}}
\def\gJ{{\mathcal{J}}}
\def\gK{{\mathcal{K}}}
\def\gL{{\mathcal{L}}}
\def\gM{{\mathcal{M}}}
\def\gN{{\mathcal{N}}}
\def\gO{{\mathcal{O}}}
\def\gP{{\mathcal{P}}}
\def\gQ{{\mathcal{Q}}}
\def\gR{{\mathcal{R}}}
\def\gS{{\mathcal{S}}}
\def\gT{{\mathcal{T}}}
\def\gU{{\mathcal{U}}}
\def\gV{{\mathcal{V}}}
\def\gW{{\mathcal{W}}}
\def\gX{{\mathcal{X}}}
\def\gY{{\mathcal{Y}}}
\def\gZ{{\mathcal{Z}}}

% Sets
\def\sA{{\mathbb{A}}}
\def\sB{{\mathbb{B}}}
\def\sC{{\mathbb{C}}}
\def\sD{{\mathbb{D}}}
% Don't use a set called E, because this would be the same as our symbol
% for expectation.
\def\sF{{\mathbb{F}}}
\def\sG{{\mathbb{G}}}
\def\sH{{\mathbb{H}}}
\def\sI{{\mathbb{I}}}
\def\sJ{{\mathbb{J}}}
\def\sK{{\mathbb{K}}}
\def\sL{{\mathbb{L}}}
\def\sM{{\mathbb{M}}}
\def\sN{{\mathbb{N}}}
\def\sO{{\mathbb{O}}}
\def\sP{{\mathbb{P}}}
\def\sQ{{\mathbb{Q}}}
\def\sR{{\mathbb{R}}}
\def\sS{{\mathbb{S}}}
\def\sT{{\mathbb{T}}}
\def\sU{{\mathbb{U}}}
\def\sV{{\mathbb{V}}}
\def\sW{{\mathbb{W}}}
\def\sX{{\mathbb{X}}}
\def\sY{{\mathbb{Y}}}
\def\sZ{{\mathbb{Z}}}

% Entries of a matrix
\def\emLambda{{\Lambda}}
\def\emA{{A}}
\def\emB{{B}}
\def\emC{{C}}
\def\emD{{D}}
\def\emE{{E}}
\def\emF{{F}}
\def\emG{{G}}
\def\emH{{H}}
\def\emI{{I}}
\def\emJ{{J}}
\def\emK{{K}}
\def\emL{{L}}
\def\emM{{M}}
\def\emN{{N}}
\def\emO{{O}}
\def\emP{{P}}
\def\emQ{{Q}}
\def\emR{{R}}
\def\emS{{S}}
\def\emT{{T}}
\def\emU{{U}}
\def\emV{{V}}
\def\emW{{W}}
\def\emX{{X}}
\def\emY{{Y}}
\def\emZ{{Z}}
\def\emSigma{{\Sigma}}

% entries of a tensor
% Same font as tensor, without \bm wrapper
\newcommand{\etens}[1]{\mathsfit{#1}}
\def\etLambda{{\etens{\Lambda}}}
\def\etA{{\etens{A}}}
\def\etB{{\etens{B}}}
\def\etC{{\etens{C}}}
\def\etD{{\etens{D}}}
\def\etE{{\etens{E}}}
\def\etF{{\etens{F}}}
\def\etG{{\etens{G}}}
\def\etH{{\etens{H}}}
\def\etI{{\etens{I}}}
\def\etJ{{\etens{J}}}
\def\etK{{\etens{K}}}
\def\etL{{\etens{L}}}
\def\etM{{\etens{M}}}
\def\etN{{\etens{N}}}
\def\etO{{\etens{O}}}
\def\etP{{\etens{P}}}
\def\etQ{{\etens{Q}}}
\def\etR{{\etens{R}}}
\def\etS{{\etens{S}}}
\def\etT{{\etens{T}}}
\def\etU{{\etens{U}}}
\def\etV{{\etens{V}}}
\def\etW{{\etens{W}}}
\def\etX{{\etens{X}}}
\def\etY{{\etens{Y}}}
\def\etZ{{\etens{Z}}}

% The true underlying data generating distribution
\newcommand{\pdata}{p_{\rm{data}}}
% The empirical distribution defined by the training set
\newcommand{\ptrain}{\hat{p}_{\rm{data}}}
\newcommand{\Ptrain}{\hat{P}_{\rm{data}}}
% The model distribution
\newcommand{\pmodel}{p_{\rm{model}}}
\newcommand{\Pmodel}{P_{\rm{model}}}
\newcommand{\ptildemodel}{\tilde{p}_{\rm{model}}}
% Stochastic autoencoder distributions
\newcommand{\pencode}{p_{\rm{encoder}}}
\newcommand{\pdecode}{p_{\rm{decoder}}}
\newcommand{\precons}{p_{\rm{reconstruct}}}

\newcommand{\laplace}{\mathrm{Laplace}} % Laplace distribution

\newcommand{\E}{\mathbb{E}}
\newcommand{\Ls}{\mathcal{L}}
\newcommand{\R}{\mathbb{R}}
\newcommand{\emp}{\tilde{p}}
\newcommand{\lr}{\alpha}
\newcommand{\reg}{\lambda}
\newcommand{\rect}{\mathrm{rectifier}}
\newcommand{\softmax}{\mathrm{softmax}}
\newcommand{\sigmoid}{\sigma}
\newcommand{\softplus}{\zeta}
\newcommand{\KL}{D_{\mathrm{KL}}}
\newcommand{\Var}{\mathrm{Var}}
\newcommand{\standarderror}{\mathrm{SE}}
\newcommand{\Cov}{\mathrm{Cov}}
% Wolfram Mathworld says $L^2$ is for function spaces and $\ell^2$ is for vectors
% But then they seem to use $L^2$ for vectors throughout the site, and so does
% wikipedia.
\newcommand{\normlzero}{L^0}
\newcommand{\normlone}{L^1}
\newcommand{\normltwo}{L^2}
\newcommand{\normlp}{L^p}
\newcommand{\normmax}{L^\infty}

\newcommand{\parents}{Pa} % See usage in notation.tex. Chosen to match Daphne's book.

\DeclareMathOperator*{\argmax}{arg\,max}
\DeclareMathOperator*{\argmin}{arg\,min}

\DeclareMathOperator{\sign}{sign}
\DeclareMathOperator{\Tr}{Tr}
\let\ab\allowbreak


% a shortcut for a spaced out vertical bar
\newcommand{\vbar}{\hspace{-0.4mm} \mathrel{|} \hspace{-0.4mm}}

\title{Gaussian Processes/Robot Arm Dynamics \\ Introduction and Methods}

\author {Jacob Lister                              \\
Department of Electrical and Computer Engineerning \\
Department of Mathematical Sciences                \\
Purdue University Fort Wayne                       \\
Fort Wayne, IN 46805, USA                          \\
\texttt{aldrjt01@pfw.edu}
}

\newcommand{\fix}{\marginpar{FIX}}
\newcommand{\new}{\marginpar{NEW}}

% to allow hyphens to appear in the author names of those cited
% https://latex.org/forum/viewtopic.php?t=3584#top
% the post from phi detailed the solution i used
% to insert a hyphen in the name of an author, instead
% of typing a -, type \nobreakhyphen
\makeatletter
\newcommand*\nobreakhyphen{\hbox{-}\nobreak\hskip\z@skip}
\makeatother

\iclrfinalcopy % Uncomment for camera-ready version, but NOT for submission.
\begin{document}

    \maketitle

    \begin{abstract}
        In this work, I analyze the inverse dynamics model of a SARCOS seven joint robotic arm.
        The technique that was used to perform this analysis was Guassian Processes Regression,
        as the functions that dictate the movement of robotic arms are complex and almost always
        non-linear. In addition, the Fully Independent Conditional Approximation was used
        due to the large size of the dataset.
    \end{abstract}

    \section{Introduction}
    
    % change all tenses in this for analysis and results
    % also expand on the results of the analysis for the analysis and results section

    In this section, I provide an overview of the inverse dynamics problem, a brief description
    of kinematics, and a brief explanation of the analysis that was performed.
    
    Inverse dynamics is used when working to model joint movement, often of a robotic arm or
    the arms and legs of an animal or person. This technique is used to determine the torque
    being applied to the joints in a system based on the position, velocity, and accelation of
    a body (or end effector in the case of a robotic arm). Using inverse dynamics, you can
    compute the torques needed to follow a certain trajectory, which provides the ability to
    simulate biological models or allows robotic arms to "plan" the path they will take when
    moving from one position to another.
    
    The specific system I am looking at is that of a SARCOS robotic arm. This arm has 7 joints,
    which means 7 torques to find. Additionally, this means that there are 7 positions, velocities,
    and accelerations as outputs. Because I was working with inverse dynamics, the positions,
    velocities, and accelerations will be used as inputs, and the torques will be used as outputs.
    This means I will be creating a mapping of 21 inputs of 7 outputs.
    
    Inverse dynamics can be performed in a variety of ways. The most straight-forward way is to
    analytically compute the torque functions of each joint in the system. This method requires
    full knowledge of the system and often becomes extremely complicated for many-jointed
    systems, so this method cannot always be used. There are also methods of estimating the
    torques using statistical methods. Due to the fact that the composition of torque functions
    is of trigonometric functions mainly, these functions end up being highly non-linear. To see
    more information about how the torque functions are composed, see the
    \textbf{\hyperref[sec:appendix]{Appendix}}. Due to the non-linearity of the torques, simple
    statistic methods such as linear regression are expected to perform poorly. Guassian Processes
    Regression is a more robust statistical method and will be the focus of this manuscript.
    Guassian Processes Regression was used to model the torques of the SARCOS arm given the 21
    inputs.
    
    I chose to work on this project with the interest of learning about a method through
    the inverse dynamics of a robotic arm can be modeled non-analytically. As I have taken an
    undergraduate robotics course, and am currently taking a graduate level robotics course, I
    have worked with forward kinematics and a small amount of inverse dynamics. The inverse
    dynamics I have worked with has been solely analytic, working out each torque value in a system
    by hand. I have been curious about learning about non-analytic methods for a while now, and
    this was the perfect oppurtunity to do so.
    
    The remaining part of this manuscript is organized as follows. Section 2 is dedicated to the
    methods that were used, Section 3 to my results and a discussion, and in Section 4 I will
    draw conclusions.
    
    \clearpage
    
    \section{Methods}
    
    In this section, I describe the data source, Guassian Processes Regression (GPR), and the
    Fully Independent Conditional Approximation.
    
    \subsection{Data Source and Software}
    
    The SARCOS dataset was sourced from the website for the textbook 
    \textit{Gaussian Processes for Machine Learning} \citet{sarcos}. The analysis was performed
    using the software MATLAB R2023A and the \texttt{Statistics and Machine Learning Toolbox}. I
    switched from Python to MATLAB dues to MATLAB's strong documentation and implmentation of
    GPR being already present, alongside the fact that I am more comfortable with MATLAB than
    Python. The dataset is publicly available and the code will be publicly available in a Github
    Repository \citep{github}.
    
    \subsection{Guassian Processes Regression}
    
    To estimate the torque functions of the SARCOS arm, I used a Guassian Processes Regression
    (GPR). GPR is a technique used in machine learning and more broadly in statistics. It is often
    used a supervised learning technique when working with non-linear systems. Because
    Guassian Processes (GPs) can interpreted as a distribution over a function space \citep{gpml},
    we can use them to create a more robust method of regression. There are some additional
    advantages to working with guassian processes \citep{scikit-gp}, some of which are detailed
    below:
    
    \begin{itemize}
        \item The prediction is probabilistic (Gaussian) so that one can compute empirical confidence intervals and decide based on those if one should refit (online fitting, adaptive fitting) the prediction in some region of interest.
        \item Versatile: different kernels can be specified. Common kernels are provided, but it is also possible to specify custom kernels.
    \end{itemize}
    
    This versatility helps GPR work well in non-linear systems. There are some downsides to GPR,
    one being the fact that it uses all samples provided to perform its prediction, so the
    computational complexity becomes difficult to manage with large datasets; additionally, GPR
    loses efficiency in higher dimensional spaces \citep{scikit-gp}, but this should not be an
    issue as I am working with a dataset with a relatively small amount of dimensions. Because of
    the fact that the dataset I am working with is large, having over 40,000 samples, an
    approximation method for GPR will be used, which is detailed in the next subsection.
    
    Now for an in-depth explanation of GPR. If we interpret a GP as being a distribution over
    a function space, we can write it as:
    
    \begin{align*}
        f(\textbf{x})                             &\sim \mathcal{GP}(m(\textbf{x}), k(\textbf{x}, \textbf{x}')) \\
        \mathbb{E}[f(\textbf{x})]                 &= m(\textbf{x})                                           \\
        \text{cov}[f(\textbf{x}), f(\textbf{x}')] &= k(\textbf{x}, \textbf{x}')
    \end{align*}
    
    where $\textbf{x}$, $\textbf{x}'$ are 2 points chosen from the set of possible inputs. Then,
    if we have a number of input points $X_{*}$, we can generate a random Guassian vector
    $\textbf{f}_{*} = f(\textbf{x}_{*}) \sim \mathcal{N}(0, K(X_{*}, X_{*}))$, where $K$ is the
    covariances matrix between the different inputs. This becomes the prior for the function
    we are trying to approximate. If we take into account noise and condition the prior on the
    observations (training data), we have \citep{gpml}:
    
    \begin{align*}
        \textbf{f}_{*} \vbar X_{*}, X, \textbf{f} \sim \mathcal{N}(K(X_{*}, X)K(X, X)^{-1} \textbf{f}, K(X_{*}, X_{*}) - K(X_{*}, X)K(X, X)^{-1} K(X, X_{*}))
    \end{align*}
    
    This means that by evaluating the mean and covariance matrices above, the function values
    $\textbf{f}_{*}$ corresponding to the inputs $X_{*}$ can be sampled from the posterior
    distribution to estimate $f(\textbf{x})$ \citep{gpml}. This provides the main basis for GPR.
    
    \subsection{Fully Independent Conditional Approximation}
    
    I was originally planning to use Projected Processes as my approximation method, but ended up
    changing to Fully Independent Conditional Approximation. This is due to a mix of personal
    and school stuff happening reducting the amount of time I had to work on this, and the fact
    the Fully Independent Conditional Approximation was built into MATLAB, which reduced the amount
    of time I had to work on this.
    
    The approximation method for GPR I used was the Fully Independent Conditional (FIC)
    Approximation. This approximation method modfies the Subset of Regressors method detailed
    in \citep{gpml} in a way that avoids the degeneracy problem as in the Subset of Regressors
    method, so it is operating on a GP still. In FIC, only a subset of the full dataset is used,
    and depending on the data being inputted into the kernel, it may either be approximated or
    calculated exactly. The kernel for the FIC approximation is shown below \citep{fic}:
    
    \begin{align*}
        \widehat{k}_{\text{FIC}}(x_{i}, x_{j} \vbar \theta, A) = (1 - \delta_{ij})\widehat{k}_{\text{SR}}(x_{i}, x_{j} \vbar \theta, A) + \delta_{ij} k(x_{i}, x_{j} \vbar \theta, A)
    \end{align*}
    
    As seen above, the kernel is approximated when the $i \neq j$, which is equivalent to the
    Subset of Regressors method, and calculated exactly when $i = j$. The FIC
    approximation of the $K(X, X \vbar \theta, A)$ matrix is given as \citep{fic_matlab}:
    
    \begin{align*}
        \widehat{K}_{\text{FIC}}(X, X  \vbar \theta, A) = \widehat{K}_{\text{SR}}(X, X  \vbar \theta, A) + \delta_{ij}(k(x_{i}, x_{j} \vbar \theta, A)) 
    \end{align*}
    
    This method was chosen over the other approximation methods for a number
    of reasons (besides what I mentioned at the beginning of this section).
    This method was chosen over the Subset of Regressors Method due to FIC being a direct
    improvement to Subset of Regressors, as it has the same computational complexity as the Subset
    of Regressors while fixing the issue with the degeneracy of the covariance matrix \citep{fic}.
    The Projected Processes Approximation was not used due to reasons mentioned earlier, and the
    Bayesian Committee Machine Method was not chosen due to it's higher computational complexity
    for similar performance. The last method that was not chosen was the Subset of Dataset method,
    and while it does have the same computational complexity, it is based on a degenerate GP, so I
    chose FIC over it for the same reasons as Subset of Regressors. Overall, the FIC Approximation
    provided a strong middle ground of the different approximation methods, leading to it being
    chosen.
    
    \section{Analysis and Results}
    
    In this discussion I discuss the results of my analysis. In \textbf{\autoref{table:table1}},
    the loss of each predicted torque function is given.
    
    \begin{table}[h] \centering
        \caption{\centering Table detailing the MSE for each torque variable (denoted $q_{i}$) estimated,
        kernel function for each variable, and basis function for each variable.}
        \vspace*{1mm}
        \begin{tabular}{cccc}
            \midrule
            \midrule
            $i$ & MSE for predicted $q_{i}$ & Kernel function for $q_{i}$ & Basis function for $q_{i}$ \\
            \midrule
            1   & 12.7200                   & Rational Quadratic          & Pure Quadratic             \\
            2   &  5.0697                   & Matern52                    & Constant                   \\
            3   &  1.3051                   & Matern52                    & Linear                     \\
            4   &  0.8692                   & Matern32                    & Quadratic                  \\
            5   &  0.0271                   & Exponential                 & None                       \\
            6   &  0.3276                   & Rational Quadratic          & Pure Quadratic             \\
            7   &  0.0962                   & Rational Quadratic          & Pure Quadratic             \\
            \midrule
            \midrule
        \end{tabular}
        \label{table:table1}
    \end{table}
    
    Overall, as the model progressed through the joint torques the MSE substantially decreased.
    This could possibly be due to the fact that earlier joints in the system (assuming earlier
    indexed joint variables are closer to the base as is standard) have a more dramatic effect
    on the location of the end effector, as small changes in $q_{1}$ or $q_{2}$ could cause large
    changes in the position, velocity, and acceleration of the arm, especially if the arm is
    large. The extremely low MSE for the later joints indicates that GPR performed well in terms
    of predicting their respective torque functions. It is possible that if I were to spend more
    time fine-tuning the optimization of the hyperparemeters or defined custom kernel functions
    that I could have achieved better performance on the earlier joint torques. Overall, the GPR
    seems to be able to well predict the torque variables for the SARCOS robotic arm.
    
    \section{Conclusions}
    
    In this project, I studied the problem of inverse dynamics, specifically on robotic arms.
    Using GPR, I was able to generate predictions for the functions of each torque variable
    of the arm using the position, velocity, and acceleration of each joint. It was found that
    GPR was better able to predict joints further from the base of the robot; future analysis
    could be performed in an attempt to provide predictions that generate a lower MSE for the
    earlier torques, possibly by defining a custom kernel for them or altering how the
    hyperparameters are optimized.
    
    \clearpage
    \renewcommand*{\bibfont}{\raggedright}
    \bibliography{analysis-results}
    \bibliographystyle{iclr2025_conference}
    
    \clearpage
    \section{Appendix}
    \label{sec:appendix}
    
    When transforming coordinate systems between different joints on a robotic arm, what ends up
    occuring is a series of rotations and translations. Transformations of this nature on a
    coordinate system are considered homogeneous transformations. Given points
    $P_{0}, P_{1} \in \mathbb{R}^{3}$ where $P{0}$ is being transformed into $P_{1}$, one can
    express rotations and translations as matrix multiplication on the points, shown below:
    
    \begin{align*}
        P_{0} = A_{\text{Rot, }x\text{, } \theta} P_{1} =
        \begin{bmatrix}
            1 & 0            &  0            & 0 \\
            0 & \cos(\theta) & -\sin(\theta) & 0 \\
            0 & \sin(\theta) &  \cos(\theta) & 0 \\
            0 & 0            &  0            & 1
        \end{bmatrix} \qquad
        P_{0} &= A_{\text{Trans, }x\text{, } d} P_{1} =
        \begin{bmatrix}
            1 & 0 & 0 & d \\
            0 & 1 & 0 & 0 \\
            0 & 0 & 1 & 0 \\
            0 & 0 & 0 & 1
        \end{bmatrix} \\
        P_{0} = A_{\text{Rot, }y\text{, } \theta} P_{1} =
        \begin{bmatrix}
             \cos(\theta) & 0 & \sin(\theta) & 0 \\
             0            & 1 & 0            & 0 \\
            -\sin(\theta) & 0 & \cos(\theta) & 0 \\
             0            & 0 & 0            & 1
        \end{bmatrix} \qquad
        P_{0} &= A_{\text{Trans, }y\text{, } d} P_{1} =
        \begin{bmatrix}
            1 & 0 & 0 & 0 \\
            0 & 1 & 0 & d \\
            0 & 0 & 1 & 0 \\
            0 & 0 & 0 & 1
        \end{bmatrix} \\
        P_{0} = A_{\text{Rot, }z\text{, } \theta} P_{1} =
        \begin{bmatrix}
            \cos(\theta) & -\sin(\theta) & 0 & 0 \\
            \sin(\theta) &  \cos(\theta) & 0 & 0 \\
            0            & 0             & 0 & 0 \\
            0            & 0             & 0 & 1
        \end{bmatrix} \qquad
        P_{0} &= A_{\text{Trans, }z\text{, } d} P_{1} =
        \begin{bmatrix}
            1 & 0 & 0 & 0 \\
            0 & 1 & 0 & 0 \\
            0 & 0 & 1 & d \\
            0 & 0 & 0 & 1
        \end{bmatrix}
    \end{align*}
    
    Additionally, there exists a procedure for defining the forward kinematics of a robotic arm
    called the Denavit-Hartenburg (DH) Procedure. In it, the following transformation is used
    when transforming between coordinate systems of the different robotic links
    (joint $i - 1$ to $i$):
    
    \begin{align*}
        A_{i - 1}^{i} &= A_{\text{Rot, }z_{i - 1}\text{, } \theta_{i}} A_{\text{Trans, }z_{i - 1}\text{, } d_{i}} A_{\text{Trans, }x_{i}\text{, } a_{i}} A_{\text{Rot, }x_{i}\text{, } \alpha_{i}} \\
                      &=
        \begin{bmatrix}
            \cos \theta_{i} & -\cos \alpha_{i} \sin \theta_{i} &  \sin \alpha_{i} \sin \theta_{i} & a_{i} \cos \theta_{i} \\
            \sin \theta_{i} &  \cos \alpha_{i} \cos \theta_{i} & -\sin \alpha_{i} \cos \theta_{i} & a_{i} \sin \theta_{i} \\
            0               &  \sin \alpha_{i}                 &  \cos \alpha_{i}                 & d_{i}                 \\
            0               &  0                               &  0                               & 1
        \end{bmatrix}
    \end{align*}
    
    where $\theta_{i}$, $d_{i}$ are possible joint variables. Additionally, their exists a nice
    property relating different joint coordinate systems to each other. Say we have a $n$ joint
    robotic arm and $0 \leq j < k \leq n$. Then the transformation matrix starting at link $j$ and
    going to link $m$ is given by:
    
    \begin{align*}
        A_{j}^{k} = A_{j}^{j + 1} A_{j + 1}^{j + 2} \cdot A_{k - 2}^{k - 1} A_{k - 1}^{k}
    \end{align*}
    
    This means when transforming from our base coordinate to the end effector frame (end point
    on the robot where a robotic gripper / arm is often located), we have:
    
    \begin{align*}
        A_{0}^{n} = A_{0}^{1} A_{1}^{2} \cdot A_{n - 2}^{n - 1} A_{n - 1}^{n}
    \end{align*}
    
    Additionally, we can obtain information about the end effector coordinate system and location
    relative to the base coordinate system via the transformation matrix. If end effector
    coordinate basis vectors are denoted as $\textbf{x}$, $\textbf{y}$, $\textbf{z}$, and the
    end effector gripper location is $\textbf{p}$, then:
    
    \begin{align*}
        A_{0}^{n} =
        \begin{bmatrix}
            \textbf{x} & \textbf{y} & \textbf{z} & \textbf{p} \\
            0          & 0          & 0          & 1
        \end{bmatrix}
    \end{align*}
    
    From the way that the DH matrices are composed, alongside their relationship to the end
    effector coordinate system, it is clear to see that the system will be highly non-linear,
    especially as more rotational joints (joints with $\theta_{i}$ as the joint variable) are
    included in the system.

\end{document}
