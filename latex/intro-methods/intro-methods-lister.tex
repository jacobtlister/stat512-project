% !TEX TS-program = pdflatex
% !TEX encoding = UTF-8 Unicode
\documentclass{article}

\newcommand{\filetitle}{intro-methods-lister}

% packages used
\usepackage{amsmath}
\usepackage{amssymb}
\usepackage{mathtools}
\usepackage{bm}
\usepackage{graphicx}
\usepackage{iclr2025_conference,times}
\usepackage{hyperref}
\usepackage{url}
\usepackage{natbib}
\usepackage{fancyhdr}

% directory where images placed in document live
\graphicspath{ {./figures/} }

% adjusting hyperref options to my liking
\hypersetup{
    colorlinks=true,
    breaklinks=true,
    urlcolor=blue,
    linkcolor=black,
    filecolor=black,
    citecolor=black,
    menucolor=black,
    urlbordercolor={1 1 1},
    linkbordercolor={1 1 1},
    filebordercolor={1 1 1},
    citebordercolor={1 1 1},
    menubordercolor={1 1 1},
    bookmarksopen=true,
    pdfpagemode=useOutlines,
    pdftitle={\filetitle},
    pdfauthor={Jacob Lister},
}

% Optional math commands from https://github.com/goodfeli/dlbook_notation.
\input{math_commands.tex}

\title{Gaussian Processes/Robot Arm Dynamics \\ Introduction and Methods}

\author {Jacob Lister                              \\
Department of Electrical and Computer Engineerning \\
Department of Mathematical Sciences                \\
Purdue University Fort Wayne                       \\
Fort Wayne, IN 46805, USA                          \\
\texttt{aldrjt01@pfw.edu}
}

\newcommand{\fix}{\marginpar{FIX}}
\newcommand{\new}{\marginpar{NEW}}

% to allow hyphens to appear in the author names of those cited
% https://latex.org/forum/viewtopic.php?t=3584#top
% the post from phi detailed the solution i used
% to insert a hyphen in the name of an author, instead
% of typing a -, type \nobreakhyphen
\makeatletter
\newcommand*\nobreakhyphen{\hbox{-}\nobreak\hskip\z@skip}
\makeatother

\iclrfinalcopy % Uncomment for camera-ready version, but NOT for submission.
\begin{document}

    \maketitle

    \begin{abstract}
        In this work, I analyze the inverse dynamics model of a SARCOS seven joint robotic arm.
        The technique that will be used to perform this analysis was Guassian Processes Regression,
        as the functions that dictate the movement of robotic arms are complex and almost always
        extremely non-linear. In addition, the Projected Processes approximation will be used due
        to the large size of the dataset.
    \end{abstract}

    \section{Introduction}
    
    % change all tenses in this for analysis and results
    % also expand on the results of the analysis for the analysis and results section

    In this section, I provide an overview of the inverse dynamics problem, a brief description
    of kinematics, and a brief explanation of the analysis that will be performed.
    
    Inverse dynamics is used when working to model joint movement, often of a robotic arm or
    the arms and legs of an animal or person. This technique is used to determine the torque
    being applied to the joints in a system based on the position, velocity, and accelation of
    a body (or end effector in the case of a robotic arm). Using inverse dynamics, you can
    compute the torques needed to follow a certain trajectory, which provides the ability to
    simulate biological models or allows robotic arms to "plan" the path they will take when
    moving from one position to another.
    
    The specific system I am looking at is that of a SARCOS robotic arm. This arm has 7 joints,
    which means 7 torques to find. Additionally, this means that there are 7 positions, velocities,
    and accelerations as outputs. Because we are working with inverse dynamics, the positions,
    velocities, and accelerations will be used as inputs, and the torques will be used as outputs.
    This means I will be creating a mapping of 21 inputs of 7 outputs.
    
    Inverse dynamics can be performed in a variety of ways. The most straight-forward way is to
    analytically compute the torque functions of each joint in the system. This method requires
    full knowledge of the system and often becomes extremely complicated for many-jointed
    systems, so this method cannot always be used. There are also methods of estimating the
    torques using statistical methods. Due to the fact that the composition of torque functions
    is of trigonometric functions mainly, these functions end up being highly non-linear. To see
    more information about how the torque functions are composed, see Appendix A (will be
    implemented in final report). Due to the non-linearity of the torques, simple statistic
    methods such as linear regression are expected to perform poorly. Guassian Processes
    Regression is a more robust statistical method and will be the focus of this manuscript.
    Guassian Processes Regression will be used to model the torques of the SARCOS arm given the 21
    inputs.
    
    I chose to work on this project with the interest of learning about a method through
    the inverse dynamics of a robotic arm can be modeled non-analytically. As I have taken an
    undergraduate robotics course, and am currently taking a graduate level robotics course, I
    have worked with forward kinematics and a small amount of inverse dynamics. The inverse
    dynamics I have worked with has been solely analytic, working out each torque value in a system
    by hand. I have been curious about learning about non-analytic methods for a while now, and
    this was the perfect oppurtunity to do so.
    
    The remaining part of this manuscript is organized as follows. Section 2 is dedicated to the
    methods that will be used.
    
    \section{Methods}
    
    In this section, I describe the data source, Guassian Processes Regression (GPR), and
    Projected Processes (PP) Approximation.
    
    \subsection{Data Source and Software}
    
    The SARCOS dataset was sourced from the website for the textbook 
    \textit{Gaussian Processes for Machine Learning} \citet{sarcos}. The analysis will be performed
    using the software Python 3.10 and the packages \texttt{numpy} and \texttt{scipy}. The dataset
    is publicly available and the code will be publicly available in a Github Repository
    \citep{github}.
    
    \subsection{Guassian Processes Regression}
    
    To estimate the torque functions of the SARCOS arm, I will use a Guassian Processes Regression
    (GPR). GPR is a technique used in machine learning and more broadly in statistics. It is often
    used a supervised learning learning technique when working with non-linear systems. Because
    Guassian Processes (GPs) can interpreted as a distribution over a function space \citep{gpml},
    we can use them to create a more robust method of regression. There are some additional
    advantages to working with guassian processes \citep{scikit-gp}, some of which are detailed
    below:
    
    \begin{itemize}
        \item The prediction is probabilistic (Gaussian) so that one can compute empirical confidence intervals and decide based on those if one should refit (online fitting, adaptive fitting) the prediction in some region of interest.
        \item Versatile: different kernels can be specified. Common kernels are provided, but it is also possible to specify custom kernels.
    \end{itemize}
    
    This versatility helps GPR work well in non-linear systems. There are some downsides to GPR,
    one being the fact that it uses all samples provided to perform its prediction, so the
    computational complexity becomes difficult to manage with large datasets; additionally, GPR
    loses efficiency in higher dimensional spaces \citep{scikit-gp}, but this should not be an
    issue as I am working with a dataset with a relatively small amount of dimensions. Because of
    the fact that the dataset I am working with is large, having over 40,000 samples, an
    approximation method for GPR will be used, which is detailed in the next subsection.
    
    \subsection{Projected Processes Approximation}
    
    The approximation method for GPR I will be using is the Projected Processes (PP) Approximation.
    This approximation method only uses a subset of the dataset to predict the function of the
    system by "absorbing" the information of the entire dataset into the subset \citep{gpml}.
    
    This method was chosen over the other approximation methods provided in the book for a number
    of reasons. This method was chosen over the Subset of Regressors Method due to PP's higher
    performance and the Bayesian Committee Machine Method was not chosen due to it's higher
    computational complexity for similar performance. The last method that was not chosen was the
    Subset of Dataset method, and while it does have the same computational complexity and similar
    performance as PP \citep{gpml}, it is based on a degenerate GP, so to maximize performance in
    predicting the torques, I chose PP over it. Overall, the Projected Processes Approximation
    provided a strong middle ground of the different approximation methods, leading to it being
    chosen.
    
    \clearpage
    \bibliography{intro-methods}
    \bibliographystyle{iclr2025_conference}

\end{document}
